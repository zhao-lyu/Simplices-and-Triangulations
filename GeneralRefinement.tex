Suppose that a domain is divided into simplices. Mesh Refinement is a procedure of mesh modification in which we divide these simplices into smaller simplices. This process can be applied recursively. Let us first introduce triangulation to help understand refinement on a simplex. Generally speaking, we can think triangulation as a subdivision of a plane into triangles. Definition below is a more formal way to take when extending to higher dimension.
    \begin{definition*}
    A triangulation of $\mathbb R^n$ is subdivision into n-dimensional simplices such that intersection of any two simplices is either empty or sharing a common face, and any face of a simplex is in the triangulation.
    \end{definition*}
    Indeed, we say that this triangulation is consistent as it is not simply subdividing of a space. Moreover, the triangulation defined here can be treated equivalently as simplicial complex as it is a finite set of simplices satisfying
    \begin{itemize}
        \item[1.] Any face of a simplex from a triangulation is also in the triangulation
        \item[2.] The intersection of any two simplices $T_1, T_2 $ in a triangulation is a face of both $T_1$ and  $T_2$ or empty
    \end{itemize}
    (Denote triangulation same as simplicial complex $\mathcal{T}$)\\
    

    We can think a refinement of a simplex $T$ as a triangulation $\mathcal{T}$ which consists of smaller pieces of simplices of same type of the simplex $T$. Now consider a refinement of a simplicial complex. Let $\mathcal{T}$ and $\mathcal{T'}$ be two different simplicial complex covering a same domain $\Omega$. This means that the domain \(\Omega = \displaystyle \bigcup({T \vert T\in \mathcal{T}}) = \bigcup({T' \vert T\in \mathcal{T'}})\). We say that $\mathcal{T'}$ is a refinement of $\mathcal{T}$ if each simplex $T\in\mathcal{T}$ is in $\mathcal{T'}$ or the triangulation of $T$ is in $\mathcal{T'}$.

    As mentioned before, we may recursively apply a refinement strategy to help simplify some problems. By recursively taking refinement process from $\mathcal{T}_0$, we have a hierarchy triangulation $\mathcal{T}_k, k\in\mathbb{N}$, where $\mathcal{T}_k$ is a refinement of $\mathcal{T}_{k-1}$. 
    \begin{definition*}
    Let $\mathcal{T}_0$ be the initial simplicial complex in $\mathbb{R}^n$ where it starts from, then we define the hierarchy triangulation $\mathcal{T}_k$ as follows
    \begin{equation*}
    \mathcal{T}_k := \bigcup\{refinement~of~simplex~T ~\vert ~T\in\mathcal{T}_{k-1}\}, \quad k\in\mathbb{N}
    \end{equation*}
    \end{definition*}

    \subsection{Consistency of Refinement}
    We also want the triangulation after applying with a refinement to be consistent. This feature is proved in section 2.4 that if either 1) any face of a simplex from this triangulation $\mathcal{T}$ is also in $\mathcal{T}$, or 2) the intersection of any two simplices in a face of both simplices.

    \subsection{Stability of Refinement}
    Introductory sentence...
    \begin{definition*}
    We say a refinement strategy is $\textbf{stable}$ if there exists a constant $C >$ 0 such that $\mu(T)< C$ for all simplices $T$.
    \end{definition*}
    
    \begin{theorem*}
    If the number of congruence classes, obtained by applying the refinement of a non degenerate simplex $T$ initially, is finite, then the refinement strategy is stable.
    \end{theorem*}
    \begin{proof}
    %Idea:\\
    %1. $T_0$ is non-degenerate, then $\mathcal{T_0}$ is non-degenerate.\\
    
    We claim that
    a refinement strategy over initial simplicial complex $T_0$ produces only non-degenerate simplicies $T$.
    
    We prove this claim by induction.
    Clearly, the base case is true since it is given that all simplices $T$ in $\mathcal{T}$ are non-degenerate. For induction, suppose simplices in simplicial complex $\mathcal{T}_k$ is non-degenerate. That is, there exists $C > 0$ such that $\mu(T) < C, ~\forall T \in\mathcal{T}_k$. Apply the refinement strategy on $\mathcal{T}_k$, and then we obtain $\mathcal{T}_{k+1} = \bigcup\{refinement~of~simplex~T ~\vert ~T\in\mathcal{T}_{k}\}, \quad k\in\mathbb{N}$. 
    %[connection???: f simplex $T_0 \in\mathcal{T}$ is non-degenerate, then simplicial complex $\mathcal{T}$ non-degenerate.]
    
    Next, we show the following fact.
    If the number of congruence classes is finite, then the number of shape measure is finite, and there exists a common bound $C > 0$ such that $C \geq \mu(\mathcal{T})$.
    
    This can be seen as follows.
    We proved that simplices in same congruence classes share the same shape measure. If we have a finite number of congruence classes, clearly we have a finite number of shape measures. When all simplices are non degenerate, we always have an upper bound for their shape measure $\mu(T)$. With the finite number of shape measures, we may set $C$ as the maximum of all upper bounds of shape measures. And therefore $C \geq \mu(\mathcal{T})$.
    
    Since $T_0$ is non-degenerate, $T_0$ is non-degenerate. Moreover, we know there exists a common bound $C$ for all shape measures since the number of congruence classes is finite. Therefore, we proved the stability.
    \end{proof}
    
    