One question in computational geometry is how to represent complicated geometric shapes as a composition of more simple shapes. For example, an area can be expressed as composition of triangles, and a volume can be written as a composition of tetrahedron. In this thesis, we discuss an important topic about such decompositions, namely the refinement of simplicial meshes.

One motivation for this topic is the numerical analysis of partial differential equations. While numerical methods simulate the behavior of a partial differential equation over some area, it is necessary  to represent the area on a computer first. This process is called triangulation, decomposing the area into simplices. Importantly, we sometimes may want to increase the resolution of this triangulation \cite{grosso1998hierarchical}. To do so, algorithms which perform simplicial mesh refinement is crucial and worthy of attention. Besides its application in particial differential equations, mesh refinement can also be applied in other field, such as solving interpolation problems \cite{moore1992simplical}.

There are two main challenges in designing such algorithms. One is to maintain the stability of these simplices. In other words, unlimited to how many times a refinement is repeated, shapes of triangles should be bounded. That is, there should not exist any degenerating triangles, which refers to those with extremely small angles. One main reason why degenerating triangles should be avoided is that these triangles lead to ill-conditioned matrices in numerical methods for partial differential equations [CITATION NEED]. Another challenge is to  preserve the consistency of the triangulation. This means that two triangles either do not touch or only touch at a common edge or vertex, and the importance of this attribute is that an algorithm is expected to refine triangles consistently and successfully. In short, these two constrains make the development of algorithms for simplicial mesh refinement a challenging problem, because not all  

In this thesis, we discuss two qualified algorithms for mesh refinement in 2 dimension. The first algorithm is called uniform refinement, classified as a global refinement \cite{bank1983some,bey2000simplicial,Bey1995}. Though its easy application and obvious qualification for stability and consistency bring a popularity to this refinement strategy, it fails to process some flexibility in refining simlicial meshes. This is because uniform refinement forces to refine the entire mesh at once. However, many applications demand a flexibility to refine the mesh only locally. Thus, another algorithm, called the newest vertex bisection, is introduced to obtain such flexibility for local mesh refinement. However, newest vertex bisection not perfect either. While newest vertex bisection of a single triangle preserves stability, preservation of consistency becomes complicated. In detail, the problem is that bisecting one triangle may depend on bisection of another. This leads to a chain reaction, which can be understood as a recursion using a stack and traversing back only when a base case is touched. Therefore, difficulties locates at determining how many triangles are part of this chain reaction, and under which conditions this chain reaction ends. In other words, we need to find out when the algorithm terminates and how long it lasts. 

In conclusion, this thesis will introduce uniform refinement and newest vertex bisection in 2 dimension for computational geometry, and prove their stability and consistency in application, and further discuss their potential limits, and possible solutions.