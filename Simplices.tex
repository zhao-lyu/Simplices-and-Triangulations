    \subsection{Simplices}
    In this section, we will introduce simplices and talk about its geometrical properties, such as diameter and volume. These information and notation are mainly from \cite{ciarlet2002finite}\cite{bey2000simplicial}.
    \noindent
    \begin{definition}
    A $k$-simplex $T \subseteq\mathbb{R}^n$ is a convex hull of $k+1$ vertices ${x}_0, \ldots, {x}_k \in \mathbb{R}^n$, which are affinely independent. We write 
    \begin{equation*}
    \begin{split}
    {T} & := [{x}_0, \cdots, {x}_k ]\\
    & := \left\{{x} = \sum\limits_{i=0}^k \lambda_i {x}_i \;\middle|\; \sum\limits_{i=0}^k \lambda_i = 1 \text{ and } 0 \leqslant \lambda_i \leqslant 1, 0 \leqslant i \leqslant k \right\}\\
    & := \left\{\lambda_0{x}_0 + \cdots + \lambda_k{x}_k \;\middle|\; \sum\limits_{i=0}^k \lambda_i = 1 \text{ and } 0 \leqslant \lambda_i \leqslant 1, 0 \leqslant i \leqslant k \right\}.
    \end{split}
    \end{equation*}
    \end{definition}
    If $k = n$, we do not address the dimension of a $k$-simplex. 2-simplices are also called \emph{triangles}, and 3-simplices are called \emph{tetrahedra}.

    \begin{definition}
    %\paragraph{Subsimplices}
    An  $l$-simplex ${S} = [{y}_0, \cdots, {y}_l]$ is called an \emph{$l$-subsimplex} of a $k$-simplex ${T} = [{x}_0, \cdots, {x}_k]$, if there exist indices $0 \leqslant {i}_0 < \ldots< {i}_l \leqslant k$ with ${y}_{i_{j}} = {x}_i$, for all $0 \leqslant j \leqslant l$.
    \end{definition}
    Since there are $k+1$ vertices in a $k$-simplex $T$, and $l+1$ vertices in $l$-subsimplices $S$, the number of $l$-subsimplices of $k$-simplices is $\binom{k+1}{l+1}$.


    %\paragraph{Simplices Equality}
    %Consider simplices $T = [{x}_0, \cdots, {x}_k]$ and $T' = [{y}_0, \cdots, {y}_k]$. We say these two simplices $T$ and $T'$ are equal, i.e. $T = T'$, if ${x}_i = {y}_i$ for $0 \leqslant i \leqslant k$. Note that the vertex ordering of simplex is fixed, so if two simplices $T$ and $T'$ denote the same set but with different vertex ordering, they are not equal; instead, we say that ${T}$ coincides with $T'$, i.e. $T \cong T'$.

    \paragraph{Simplices under Affine Transformation}\mbox{}\\
    Let $F$ be an affine transformation. 
    Instead of taking a single variable $x\in\mathbb{R}^n$ for affine transformation, we can take a subset S $\subseteq\mathbb{R}^n$, which contains $x\in\mathbb{R}^n$. Then the transformed set $S'$ is
    \begin{equation*}
    S\textprime= \left\{{F}(x) ~|~ x\in S \right\}.
    \end{equation*}
    Similarly, if we regard a k-dimensional simplex ${T} = [{x_0, \cdots, x_k}]$ as a subset of $\mathbb{R}^n$ then the image of ${T}$ under affine transformation is 
    \begin{equation*}
    \begin{split}
    {F}({T}) = \big[{F}(x_0), \cdots , {F}(x_k)\big].
    \end{split}
    \end{equation*}
    We can see that $F(T)$ is still a $k$-dimensional simplex. Let us write $T' = F(T)$. We might be curious about the relationship between simplices $T$ and $T'$. 
    %Since vertices of a simplex are in a specific given order, so different vertex ordering leads to different simplices. Therefore, there exists a unique affine transformation such that $T = T'$. 
    An important property of simplices is congruence.

    \begin{definition}
    Two simplices $T$, $T'$ are defined to be congruent if they can be obtained from each other by rotation, mirroring, scaling, and translation, i.e. if there exists a scaling factor c $\in\mathbb{R}^{+}$, a translation vector $v\in\mathbb{R}^n$, and an orthogonal matrix $Q\in\mathbb{R}^{n\times n}$ such that
    \begin{equation*}
    T' = v + cQT.
    \end{equation*}
    \end{definition}
    \noindent
    %When the two simplices $T$ and $T'$ have same vertices but with different vertex ordering, we can translate it as $T' = F(T)$. Based on how we define the affine transformation, it is not hard to see that ${T} \cong {T}\textprime$. 
    Then we say that ${T}$ and ${T}\textprime$ are in a same congruence class. Formally, a congruent class is an equivalence class of simplices under the congruence relation.
    We write $T\cong T'$ if $T$ and $T'$ are simplices in the same congruence class. When $T$ and $T'$ share same shape but not necessarily same size, we say $T$ is similar to $T'$, and write $T\sim T'$.
    