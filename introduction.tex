One question in computational geometry is how to represent complicated geometric shapes as a composition of more simple shapes. For example, an area can be expressed as a composition of triangles, and a volume can be written as a composition of tetrahedra. In this thesis, we discuss an important topic about such decompositions, namely the refinement of simplicial meshes.

One motivation for this topic is the numerical analysis of partial differential equations. While numerical methods simulate the behavior of a partial differential equation over some area, it is necessary  to represent the area on a computer first. This process is called triangulation, decomposing the area into simplices. Importantly, we sometimes may want to increase the resolution of this triangulation. To do so, algorithms which perform simplicial mesh refinement are crucial and worthy of attention. Besides its application in partial differential equations \cite{grosso1998hierarchical}, mesh refinement can also be applied in other fields, such as solving interpolation problems \cite{moore1992simplical}.

There are two main challenges in designing such algorithms. One is to maintain the stability of these simplices. In other words, unlimited to how many times a refinement is repeated, shapes of triangles should be bounded. That is, there should not exist any degenerating triangles, which refer to those with extremely small angles. One main reason why degenerating triangles should be avoided is that these triangles lead to ill-conditioned matrices in numerical methods for partial differential equations \cite{bank1989conditioning}. Another challenge is to  preserve the consistency of the triangulation. This means that two triangles either do not touch or only touch at a common edge or vertex, and the importance of this attribute is that an algorithm is expected to refine triangles consistently and successfully. In short, these two constraints make the development of algorithms for simplicial mesh refinement a challenging problem.

In this thesis, we discuss two applicable algorithms for mesh refinement in two dimensions. The first algorithm is called uniform refinement, one popular global refinement, of which the refinement is done all at once \cite{bank1983some,Bey1995,bey2000simplicial}. Though its easy application and obvious qualification for stability and consistency bring a popularity to this refinement strategy, it fails to provide some flexibility in refining simplicial meshes since uniform refinement is applied to whole simplices simultaneously. This is because uniform refinement forces to refine the entire mesh at once. However, many applications demand the flexibility to refine the mesh only locally. Thus, another algorithm, called the newest vertex bisection, is introduced to obtain such flexibility for local mesh refinement. However, newest vertex bisection is not perfect either. While newest vertex bisection of a single triangle preserves stability, preservation of consistency becomes complicated. In detail, the problem is that bisecting one triangle may depend on bisection of another. This leads to a chain reaction, which can be understood as a recursion using a stack and traversing back only when a base case is touched. Therefore, the difficulty is to determine how many triangles are part of this chain reaction, and under which conditions this chain reaction ends. In other words, it is not trivial whether the algorithm terminates and how long it lasts. 

In conclusion, this thesis will introduce uniform refinement and newest vertex bisection in two dimensions for computational geometry, and prove their stability and consistency in application, and further discuss their potential limits, possible solutions and their application in three dimensions and difficulties.

