% !TEX root = Prokect199.tex
\subsection{Simplicial Complex}
    \begin{definition}
    A $\emph{simplicial complex} ~~\mathcal T$ in $\mathbb{R}^n$ is a finite set of simplices in $\mathbb{R}^n$ that satisfies the following conditions:
    \begin{enumerate}[label =\arabic*.]
      \item Any subsimplex of a simplex from $\mathcal{T}$ is also in $\mathcal{T}$.
      \item The intersection of any two simplices ${T}_1, {T}_2 \in \mathcal{T}$ is a face of both ${T}_1$ and  ${T}_2$.
    \end{enumerate}
    \begin{figure}[b]
    \centering
    \captionsetup{justification=centering}
    \begin{tikzpicture}[scale = 0.6]
    \coordinate (x) at (-6,0);
    \coordinate (y) at (-2,0);
    \coordinate (z) at (-4,2);
    \draw[thick] (x) -- (y) -- (z) -- cycle;

    \coordinate (a) at (0,0);
    \coordinate (b) at (4,0);
    \coordinate (c) at (1,2);
    \coordinate (d) at (3.6,2.5);
    \draw[thick] (a) -- (b) -- (c) -- cycle;
    \draw[thick] (d) -- (b) -- (c) -- cycle;

    \coordinate (a') at (6,0);
    \coordinate (b') at (9,0);
    \coordinate (c') at (7,3);
    \coordinate (d') at (12,1);
    \draw[thick] (a') -- (b') -- (c') -- cycle;
    \tkzDefMidPoint(b',c') \tkzGetPoint{bc}
    \fill[black!20, draw=black, thick] (bc) circle (1pt) node[black, above right] {$x$};
    \coordinate (e') at ($(bc)!2!0:(b')$);
    \draw[thick] (bc) -- (d') -- (e') -- cycle;
    
    \coordinate (q) at (14,0);
    \coordinate (w) at (18,0);
    \coordinate (e) at (15,4);
    \coordinate (r) at (18,2);
    \tkzDefMidPoint(q,r) \tkzGetPoint{mid}
    \draw[thick] (q) -- (w) -- (r) -- (e) -- cycle;
    \draw[thick] (q) -- (r);
    \draw[thick] (e) -- (mid);
    \fill[black!20, draw=black, thick] (mid) circle (1pt) node[black, above right] {$y$};

    \end{tikzpicture}
    %\includegraphics[width=60mm]{SimplicialComplex}
    \caption{Simplicial complex(left); Not simplicial complex(right)\\Two simplices in Figure\ref{Fig2} on the right are not simplicial complex because their intersection $x$ and $y$ are not shared. We call such nodes like $x$ and $y$ hanging node}%cite???
    \label{Fig2}
    \end{figure}
    \end{definition}
    In other words, the first condition asks $\mathcal{T}$ to be closed under taking subsimplices, and the second condition asks that the intersection of any two simplices is either a common subsimplex or empty because the empty set is a face of every simplex. Examples in 2D are shown in Figure \ref{Fig2}. 
    %Notice that two simplices in Figure\ref{Fig2} on the right are not simplicial complex because their intersection $x$ and $y$ are not shared. We call such nodes like $x$ and $y$ hanging node.

    Any subset ${T\textprime}\in{T}$ that is itself a simplicial complex is called a $\emph{subcomplex}$ of ${T}$. We say that a $\emph{simplicial $k$-complex}$ $\mathcal T$ is a simplicial complex where the largest dimension of any simplex in $\mathcal T$ is ${k}$. So a simplicial 2-complex must not contain tetrahedra or higher dimension simplices. The 0-complex of ${T}$ is called a $\textit{vertex set}$ of ${T}$. We can also think of a simplicial complex as a space with a triangulation, which is the division of a surface or a plane polygon into a set of 2-simplices. 

    \paragraph{Simplicial Complex under Affine Transformation---\textbf{DELETE???????}}\mbox{}\\
    Extending further from simplex under affine transformation, now we know that simplicial complex is just a finite set of simplices. Therefore, we can define the Transformed Simplicial Complex $F(\mathcal{T})$ as follows
    \begin{equation*}
    F(\mathcal{T}) := \{ \; F(T) \;\vert\; T \in \mathcal{T} \; \}
    \end{equation*}
    If $\mathcal{T}$ is consistent, then $F(\mathcal{T})$ is also consistent by inheriting this property from $\mathcal{T}$.

    \paragraph{Shape Measure of Simplicial Complex}\mbox{}\\
    Recall the definition of the shape measure of a simplex.
    Now consider a simplicial complex $\mathcal{T}$, we define the geometric shape measure $\mu(\mathcal{T})$ as follows,
    \begin{equation*}
    \mu(\mathcal{T}) = \max_{T \in \mathcal{T}} \mu(T)
    \end{equation*}
    By definition, we see that the shape measure of a simplicial complex $\mathcal{T}$ is the supremum of the set of shape measures of all simplices $T\in\mathcal{T}$. If the largest shape measure of a simplex in this simplicial complex is bounded, then none of the simplices in $\mathcal{T}$ are degenerate. %In other words, if simplex $T_0 \in\mathcal{T}$ is non-degenerate, then the simplicial complex $\mathcal{T}$ non-degenerate.
    %[Correct?? Pf needed???]